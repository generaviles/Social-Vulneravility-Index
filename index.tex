% Options for packages loaded elsewhere
\PassOptionsToPackage{unicode}{hyperref}
\PassOptionsToPackage{hyphens}{url}
\PassOptionsToPackage{dvipsnames,svgnames,x11names}{xcolor}
%
\documentclass[
  letterpaper,
  DIV=11,
  numbers=noendperiod]{scrreprt}

\usepackage{amsmath,amssymb}
\usepackage{iftex}
\ifPDFTeX
  \usepackage[T1]{fontenc}
  \usepackage[utf8]{inputenc}
  \usepackage{textcomp} % provide euro and other symbols
\else % if luatex or xetex
  \usepackage{unicode-math}
  \defaultfontfeatures{Scale=MatchLowercase}
  \defaultfontfeatures[\rmfamily]{Ligatures=TeX,Scale=1}
\fi
\usepackage{lmodern}
\ifPDFTeX\else  
    % xetex/luatex font selection
\fi
% Use upquote if available, for straight quotes in verbatim environments
\IfFileExists{upquote.sty}{\usepackage{upquote}}{}
\IfFileExists{microtype.sty}{% use microtype if available
  \usepackage[]{microtype}
  \UseMicrotypeSet[protrusion]{basicmath} % disable protrusion for tt fonts
}{}
\makeatletter
\@ifundefined{KOMAClassName}{% if non-KOMA class
  \IfFileExists{parskip.sty}{%
    \usepackage{parskip}
  }{% else
    \setlength{\parindent}{0pt}
    \setlength{\parskip}{6pt plus 2pt minus 1pt}}
}{% if KOMA class
  \KOMAoptions{parskip=half}}
\makeatother
\usepackage{xcolor}
\setlength{\emergencystretch}{3em} % prevent overfull lines
\setcounter{secnumdepth}{5}
% Make \paragraph and \subparagraph free-standing
\makeatletter
\ifx\paragraph\undefined\else
  \let\oldparagraph\paragraph
  \renewcommand{\paragraph}{
    \@ifstar
      \xxxParagraphStar
      \xxxParagraphNoStar
  }
  \newcommand{\xxxParagraphStar}[1]{\oldparagraph*{#1}\mbox{}}
  \newcommand{\xxxParagraphNoStar}[1]{\oldparagraph{#1}\mbox{}}
\fi
\ifx\subparagraph\undefined\else
  \let\oldsubparagraph\subparagraph
  \renewcommand{\subparagraph}{
    \@ifstar
      \xxxSubParagraphStar
      \xxxSubParagraphNoStar
  }
  \newcommand{\xxxSubParagraphStar}[1]{\oldsubparagraph*{#1}\mbox{}}
  \newcommand{\xxxSubParagraphNoStar}[1]{\oldsubparagraph{#1}\mbox{}}
\fi
\makeatother


\providecommand{\tightlist}{%
  \setlength{\itemsep}{0pt}\setlength{\parskip}{0pt}}\usepackage{longtable,booktabs,array}
\usepackage{calc} % for calculating minipage widths
% Correct order of tables after \paragraph or \subparagraph
\usepackage{etoolbox}
\makeatletter
\patchcmd\longtable{\par}{\if@noskipsec\mbox{}\fi\par}{}{}
\makeatother
% Allow footnotes in longtable head/foot
\IfFileExists{footnotehyper.sty}{\usepackage{footnotehyper}}{\usepackage{footnote}}
\makesavenoteenv{longtable}
\usepackage{graphicx}
\makeatletter
\newsavebox\pandoc@box
\newcommand*\pandocbounded[1]{% scales image to fit in text height/width
  \sbox\pandoc@box{#1}%
  \Gscale@div\@tempa{\textheight}{\dimexpr\ht\pandoc@box+\dp\pandoc@box\relax}%
  \Gscale@div\@tempb{\linewidth}{\wd\pandoc@box}%
  \ifdim\@tempb\p@<\@tempa\p@\let\@tempa\@tempb\fi% select the smaller of both
  \ifdim\@tempa\p@<\p@\scalebox{\@tempa}{\usebox\pandoc@box}%
  \else\usebox{\pandoc@box}%
  \fi%
}
% Set default figure placement to htbp
\def\fps@figure{htbp}
\makeatother
% definitions for citeproc citations
\NewDocumentCommand\citeproctext{}{}
\NewDocumentCommand\citeproc{mm}{%
  \begingroup\def\citeproctext{#2}\cite{#1}\endgroup}
\makeatletter
 % allow citations to break across lines
 \let\@cite@ofmt\@firstofone
 % avoid brackets around text for \cite:
 \def\@biblabel#1{}
 \def\@cite#1#2{{#1\if@tempswa , #2\fi}}
\makeatother
\newlength{\cslhangindent}
\setlength{\cslhangindent}{1.5em}
\newlength{\csllabelwidth}
\setlength{\csllabelwidth}{3em}
\newenvironment{CSLReferences}[2] % #1 hanging-indent, #2 entry-spacing
 {\begin{list}{}{%
  \setlength{\itemindent}{0pt}
  \setlength{\leftmargin}{0pt}
  \setlength{\parsep}{0pt}
  % turn on hanging indent if param 1 is 1
  \ifodd #1
   \setlength{\leftmargin}{\cslhangindent}
   \setlength{\itemindent}{-1\cslhangindent}
  \fi
  % set entry spacing
  \setlength{\itemsep}{#2\baselineskip}}}
 {\end{list}}
\usepackage{calc}
\newcommand{\CSLBlock}[1]{\hfill\break\parbox[t]{\linewidth}{\strut\ignorespaces#1\strut}}
\newcommand{\CSLLeftMargin}[1]{\parbox[t]{\csllabelwidth}{\strut#1\strut}}
\newcommand{\CSLRightInline}[1]{\parbox[t]{\linewidth - \csllabelwidth}{\strut#1\strut}}
\newcommand{\CSLIndent}[1]{\hspace{\cslhangindent}#1}

\KOMAoption{captions}{tableheading}
\makeatletter
\@ifpackageloaded{bookmark}{}{\usepackage{bookmark}}
\makeatother
\makeatletter
\@ifpackageloaded{caption}{}{\usepackage{caption}}
\AtBeginDocument{%
\ifdefined\contentsname
  \renewcommand*\contentsname{Table of contents}
\else
  \newcommand\contentsname{Table of contents}
\fi
\ifdefined\listfigurename
  \renewcommand*\listfigurename{List of Figures}
\else
  \newcommand\listfigurename{List of Figures}
\fi
\ifdefined\listtablename
  \renewcommand*\listtablename{List of Tables}
\else
  \newcommand\listtablename{List of Tables}
\fi
\ifdefined\figurename
  \renewcommand*\figurename{Figure}
\else
  \newcommand\figurename{Figure}
\fi
\ifdefined\tablename
  \renewcommand*\tablename{Table}
\else
  \newcommand\tablename{Table}
\fi
}
\@ifpackageloaded{float}{}{\usepackage{float}}
\floatstyle{ruled}
\@ifundefined{c@chapter}{\newfloat{codelisting}{h}{lop}}{\newfloat{codelisting}{h}{lop}[chapter]}
\floatname{codelisting}{Listing}
\newcommand*\listoflistings{\listof{codelisting}{List of Listings}}
\makeatother
\makeatletter
\makeatother
\makeatletter
\@ifpackageloaded{caption}{}{\usepackage{caption}}
\@ifpackageloaded{subcaption}{}{\usepackage{subcaption}}
\makeatother

\usepackage{bookmark}

\IfFileExists{xurl.sty}{\usepackage{xurl}}{} % add URL line breaks if available
\urlstyle{same} % disable monospaced font for URLs
\hypersetup{
  pdftitle={Índice de Vulnerablidad Social},
  pdfauthor={Bioing. Betsaida López, Bioing. Laura González, Dr.~Gener Avilés R},
  colorlinks=true,
  linkcolor={blue},
  filecolor={Maroon},
  citecolor={Blue},
  urlcolor={Blue},
  pdfcreator={LaTeX via pandoc}}


\title{Índice de Vulnerablidad Social}
\author{Bioing. Betsaida López, Bioing. Laura González, Dr.~Gener Avilés
R}
\date{2025-01-27}

\begin{document}
\maketitle

\renewcommand*\contentsname{Table of contents}
{
\hypersetup{linkcolor=}
\setcounter{tocdepth}{2}
\tableofcontents
}

\bookmarksetup{startatroot}

\chapter*{Prefacio}\label{prefacio}
\addcontentsline{toc}{chapter}{Prefacio}

\markboth{Prefacio}{Prefacio}

This is a Quarto book.

To learn more about Quarto books visit
\url{https://quarto.org/docs/books}.

\bookmarksetup{startatroot}

\chapter{Resumen}\label{resumen}

In summary, this book has no content whatsoever.

\bookmarksetup{startatroot}

\chapter{Introducción}\label{introducciuxf3n}

El Índice de Vulnerabilidad Social (IVS) es una medida compuesta
utilizada para evaluar el grado de vulnerabilidad de diferentes
poblaciones ante riesgos socioeconómicos y ambientales. Se calcula a
partir de diversas dimensiones, como la pobreza, la educación, el acceso
a servicios de salud y la infraestructura. Este índice es ampliamente
utilizado en estudios epidemiológicos, políticas públicas y estrategias
de intervención comunitaria, permitiendo identificar poblaciones en
mayor riesgo y diseñar estrategias para reducir su vulnerabilidad.

Dado que la vulnerabilidad social es un fenómeno complejo, la
actualización constante del IVS y su adaptación a las realidades
regionales y locales son esenciales para mejorar las condiciones de vida
de los sectores más desfavorecidos y promover un desarrollo equitativo y
sostenible.

Este documento detalla el modelo empleado, los componentes, indicadores
y variables utilizadas en la elaboración del IVS, proporcionando una
comprensión integral de su estructura y aplicación.

Para el caso del Índice de vulnerabilidad social, es importante tener en
cuenta los trabajos de {[}1{]} y {[}2{]}.

\bookmarksetup{startatroot}

\chapter{Antecedentes}\label{antecedentes}

La vulnerabilidad como fenómeno social, implica la presencia de una
condición de riesgo que un individuo o familia padece como resultado de
la acumulación de desventajas sociales {[}3{]}. En México, la asistencia
social es una responsabilidad del Estado y un derecho de todos los
mexicanos, y se han realizado esfuerzos por medio de diversas
organizaciones para regularla.

El IVS del Sistema Nacional para el Desarrollo Integral de la Familia
(DIF) es un componente clave de la Fórmula para la Distribución del
Presupuesto del Ramo 33 Fondo V.i. Surge como una respuesta a la
necesidad de distribuir los recursos del país en función de criterios
pertinentes y correspondientes a la problemática que atiende el DIF
{[}3{]}.

Mediante el IVS, se mide la magnitud de la población objetivo que es
atendida con los programas de la Estrategia Integral de Asistencia
Social Alimentaria (EIASA). El primer cálculo del IVS fue en 2002 y la
fórmula incorporaba distintos componentes que competen a la asistencia
social, con una base de cálculo específica y diversas variables. Desde
entonces, la fórmula y sus cálculos se han continuado actualizando,
utilizando las fuentes de información recientes {[}3{]}. En este
sentido, el IVS se basa en variables sociales, económicas, educativas y
de salud, considerando dos grandes tipos de vulnerabilidad: la familiar
e individual, y la infantil {[}3{]}. Este índice permite a los Sistemas
Estatales DIF analizar la vulnerabilidad social en cada estado,
destacando problemas específicos y su intensidad. Facilita la
focalización de las acciones en zonas del país con mayor incidencia de
vulnerabilidad, identificando aquellos sectores que requieren de
atención prioritaria. La fórmula que valora el IVS permite establecer en
mayor medida una situación más objetiva, equitativa y transparente en la
distribución de los recursos, respetando la diversidad regional y
estatal, multicultural, étnica, de género y generacional {[}4{]}.

Otro ejemplo de la aplicación del Índice de Vulnerabilidad fue el
reporte del Índice de Vulnerabilidad Municipal a la COVID-19, cuyo
objetivo fue identificar los municipios más vulnerables ante la amenaza
de la enfermedad y entender qué características están más ligadas a esta
vulnerabilidad. Para ello, se consideraron aspectos clave como el
desarrollo social, la economía y la salubridad {[}5{]}.

Por otro lado, diversas organizaciones han desarrollado índices que
abordan distintos aspectos de la vulnerabilidad social, adaptándose a
las problemáticas específicas que cada una atiende por medio de diversos
criterios o indicadores. Por ejemplo, el Índice de Marginación (IM) del
Consejo Nacional de Población (CONAPO) evalúa el grado de marginación de
las poblaciones en México. El Índice de Desarrollo Humano (IDH) del
Programa de las Naciones Unidas para el Desarrollo (PNUD) mide el
progreso de los países en dimensiones como la salud, la educación y el
nivel de vida. Por su parte, el Índice de Rezago Social (IRS) del
Consejo Nacional de Evaluación de la Política de Desarrollo Social
(CONEVAL) permite identificar las carencias sociales en las entidades
federativas, municipios y localidades de México {[}6{]}.

\bookmarksetup{startatroot}

\chapter{Método}\label{muxe9todo}

\bookmarksetup{startatroot}

\chapter{Resultados}\label{resultados}

\bookmarksetup{startatroot}

\chapter*{Referencias}\label{referencias}
\addcontentsline{toc}{chapter}{Referencias}

\markboth{Referencias}{Referencias}

\phantomsection\label{refs}
\begin{CSLReferences}{0}{0}
\bibitem[\citeproctext]{ref-spielman2020evaluating}
\CSLLeftMargin{{[}1{]} }%
\CSLRightInline{S. E. Spielman \emph{et al.}, {``Evaluating social
vulnerability indicators: Criteria and their application to the social
vulnerability index,''} \emph{Natural hazards}, vol. 100, pp. 417--436,
2020.}

\bibitem[\citeproctext]{ref-mah2023social}
\CSLLeftMargin{{[}2{]} }%
\CSLRightInline{J. C. Mah, J. L. Penwarden, H. Pott, O. Theou, and M. K.
Andrew, {``Social vulnerability indices: A scoping review,''} \emph{BMC
public health}, vol. 23, no. 1, p. 1253, 2023.}

\bibitem[\citeproctext]{ref-sndif20017formula}
\CSLLeftMargin{{[}3{]} }%
\CSLRightInline{S. N. para el Desarrollo Integral de la Familia,
{``Índice de vulnerabilidad social.''} Dirección de Alimentación y
Desarrollo Comunitario, 2017.}

\bibitem[\citeproctext]{ref-sndif2002formula}
\CSLLeftMargin{{[}4{]} }%
\CSLRightInline{S. N. para el Desarrollo Integral de la Familia,
{``Fórmula de distribución del índice de vulnerabilidad social.''}
Dirección de Alimentación y Desarrollo Comunitario, 2002.}

\bibitem[\citeproctext]{ref-SierraAlcocer2020Indice}
\CSLLeftMargin{{[}5{]} }%
\CSLRightInline{R. S. Alcocer, G. G. Farías, and P. L. Ramírez,
{``Índice de vulnerabilidad municipal a COVID-19.''} CONAHCYT, 2020.}

\bibitem[\citeproctext]{ref-tecscience2022vulnerabilidad}
\CSLLeftMargin{{[}6{]} }%
\CSLRightInline{G. M. García, {``La vulnerabilidad social: Criterios
para su medición.''} Tecnológico de Monterrey, 2022. Available:
\url{https://tecscience.tec.mx/es/divulgacion-ciencia/la-vulnerabilidad-social-criterios-para-su-medicion/}}

\end{CSLReferences}




\end{document}
